% !TEX root = English.tex %指定主文件
\ifx\collections\undefined
% !TEX program = xelatex %指定编译方式xelatex。
% !BIB program = biber %指定bib数据后台处理程序biber。
\documentclass[11pt,a4paper,UTF8,titlepage]{ctexrep} %指定ctex的文档类,设置基本字号为11pt,a4大小,使用UTF-8编码保存,指定\maketitle生成单独的标题页。

\usepackage{syntonly}
%\syntaxonly %用来快速编译以排查错误,不生成DVI或PDF。
\usepackage[style=gb7714-2015]{biblatex} %调用biblatex宏包,设置参考文献样式(符合中文文献著录标准GB/T 7714-2015的样式),使用默认的后端程序biber(其支持更好,包括UTF-8等)放弃使用[backend=bibtex](只支持ascii编码)。
\addbibresource{resources/reference.bib} %加载参考文献数据库。
\usepackage{makeidx}
\makeindex %开启索引收集
\usepackage[margin=1in]{geometry} %调用geometry宏包,设置周围页边距为1英寸(为电子档设计,非打印)。
\usepackage{xcolor} %调用xcolor宏包,以支持扩展生成颜色。
\usepackage{fontspec} %调用fontspec宏包以更改西文字体族。
\setmainfont{Source Serif Pro}
\setsansfont{Source Sans Pro}
\setmonofont{Source Code Pro}
\usepackage{xeCJK} %调用xeCJK宏包以更改中文字体族。
\xeCJKsetup{AutoFakeSlant=true}  %设置xeCJK选项-伪斜体(需置于字体设置之前)。
\setCJKmainfont{思源宋体}
\setCJKsansfont{思源黑体}
\setCJKmonofont{思源等宽}
\usepackage{graphicx} %调用graphicx宏包,以支持插图。
\graphicspath{{resources/images/},{resources/images/FollowMe/}} %加载图片路径。
\usepackage{caption} %调用caption,支持不带编号的标题。
\usepackage{wrapfig} %调用wrapfig,支持图文排版。
\usepackage{subfigure} %调用subfigure宏包进行图片图片排版。
\usepackage{tikz,tikz-qtree} %调用tikz及其扩展宏包,以支持画图。
\usepackage[subfigure]{tocloft} %tocloft与subfigure宏包冲突,不能简单调用,tocloft需设置参数。
\usepackage{tocbibind} %调用宏包以添加目录本身和参考文献进目录中。
\usepackage{multicol} %调用multicol宏包以支持多栏排版。
\usepackage[toc]{multitoc} %调用multitoc宏包,设置toc目录页,默认双栏排版。
\usepackage{enumitem} %调用emuitem,以设置列表环境。
\usepackage{multirow} %调用multirow,以支持纵向合并列表。
%重订制目录命令。
\renewcommand{\tableofcontents}%
  {\chapter{\contentsname}%
  \@mkboth{\MakeUppercase\contentsname}{\MakeUppercase\contentsname}%
  \@makeschapterhead{\sourcecodename}%
  \@starttoc{toc}%
}
\usepackage{fancyhdr} %调用宏包,以设置页眉,统一格式:标题在左,页码在右。
\pagestyle{fancy}
\fancyhf{}
\fancyhead[LO]{\sffamily \rightmark}
\fancyhead[LE]{\sffamily \leftmark}
\fancyhead[ROE]{\bfseries \thepage}
\fancyfoot[COE]{\ttfamily \href{https://mister-kin.github.io}{个人博客:https://mister-kin.github.io}}
%定义intro环境。
\newenvironment{intro}{\narrower\sffamily}{\par\vspace*{2ex plus 2.5ex minus 1.5ex}}
\usepackage{listings} % 指定listings,订制代码排版环境
\lstset{
    basicstyle      = \ttfamily,                          % 基本代码指定等宽字体
    keywordstyle    = \bfseries,                          % 关键字指定加粗
    commentstyle    = \ttfamily\slshape\color{gray},      % 注释指定灰色等宽斜体
    stringstyle     = \ttfamily,                          % 字符串指定等宽字体
    %numbers        = left,                               % 行号的位置在左边,启用后不方便复制代码
    %numberstyle    = \ttfamily,                          % 行号等宽字体
    %xleftmargin    = \parindent,                         % 代码左边框起始位置(启用行号时建议启用这个)
    %frame          = trBL,                               % 代码框类型,t下,r右,b下,l左,大写时为两条线。
    %frameround     = fttt,                               % 控制代码框是否为圆角
    frameshape      = {{ryrynyyyy}{yny}{yny}{ryrynyyyy}}, % 控制边框样式,上下边是每三个字母段控制一条边框。
    backgroundcolor = \color{gray!5},                     % 代码框背景颜色:5%的灰色
    breaklines      = true,                               % 代码过长时则换行
    gobble          = 8,                                  % 去掉代码前的缩进
}
\usepackage{hyperref} %调用hyperref宏包
\hypersetup{
    colorlinks=true, %设置超链接文件带颜色
    bookmarks=true, %生成书签
    bookmarksopen=true, %书签展开
    bookmarksnumbered=true, %书签带章节编号
    CJKbookmarks=true, %cjk必设参数
    unicode, %utf-8编码必设参数
    pdftitle=标题, %设置PDF文件属性标题
    pdfauthor=Mr. Kin, %设置PDF文件属性作者
    pdfstartview=FitH %默认适合宽度显示
}

    \title{\hypertarget{title}{\textbf{标题}}}
    \addcontentsline{toc}{chapter}{标题页}
    \author{Written by Mr. Kin}
    \date{创建于2020.1.28,修改于\number\year.\number\month.\number\day}

\begin{document}
    \phantomsection %确保目录中的超链接指向正确的页码
    \pdfbookmark[1]{标题页}{title} %添加标题页书签
    \pagenumbering{Roman} %大写罗马样式页码
    \maketitle %生成标题页
    \pagenumbering{roman} %小写罗马样式页码
    {\centering \tableofcontents} %生成目录页
    \clearpage %新建页,分离上下两个样式页码的效果
    \pagenumbering{arabic} %阿拉伯样式页码
    \fi

    \chapter{电力专业英语}
    \section{Fundamentals of Electronic Circuits}
    \subsection{Circuit Theory}
    \begin{multicols}{2}
        \begin{itemize}
            \item element n.成分,元件
            \item interconnect vt.使互相连接
            \item node n.节点
            \item branch n.树枝,分枝,分部,支流,支脉 v.出现分岐
            \item loop n.环,循环,线圈(绳),弯曲部分,回路,回线 vt.使成环,使成圈,以环相连 vi.打环,翻筋斗
            \item topology n.拓扑,布局,拓扑学
            \item configuration n.构造,结构,配置,外形
            \item terminal n.终点站,终端,接线端
            \item resistor n.[电]电阻器
            \item independent n.独立自主者,无党派者 adj.独立自主的,不受约束的
            \item series n.连续,系列,级数,串联
            \item parallel adj.平行的,相同的,类似的,并联的 n.平行线,平行面,相似物 v.相应,平行
            \item impedance n.[电/物]阻抗,[物]全电阻
            \item theorem n.定理,法则(数)
        \end{itemize}
    \end{multicols}

    \subsection{Analog and Digital Circuits}
    \begin{multicols}{2}
        \begin{itemize}
            \item analog n.(AmE)类似物;模拟 adj.(AmE)模拟的,指针式的\\analogue n./adj.(BrE)同上
            \item digital n.数字,数字式 adj.数字的,数位的
            \item thermometer n.温度计,体温计
            \item Fahrenheit n.华氏温度计 adj.华氏温度计的
            \item drum n.鼓,鼓声,鼓形圆筒 vt.打鼓奏 vi.击鼓,做鼓声
            \item discrete adj.不连续的,离散的
            \item original adj.最初的,原始的,独创的,新颖的 n.原物,原作
            \item remote adj.遥远的,偏僻的,细微的
            \item bulb n.鳞茎,球形物
            \item Morse code 莫尔斯电码
            \item pulse n.脉搏,脉冲
            \item buzzer n.蜂鸣器,信号手,嗡嗡作声的东西
            \item manipulate vt.(熟练地)操作,使用(机器),操纵(人或市场、市场),利用,应付
            \item destination n.目的地,[计]目的文件,目的单元格
            \item humidity n.湿气,潮湿,湿度
            \item comparator n.比较仪
            \item trigger vt.引发,引起,触发 vi.转义,换车 n.扳机
            \item sequence n.次序,顺序,序列
            \item parallel n.平行的,相同的,类似的,并联的 n.平行线,平行面,相似物 v.相应,平行
            \item serial adj.连续的,串行的,顺次
            \item decoder n.解码器
            \item reassemble vt.重新召集 vi.重新集合
        \end{itemize}
    \end{multicols}

    \subsection{Three-Phase Circuits}
    \begin{multicols}{2}
        \begin{itemize}
            \item transformer n.变压器
            \item single-phase 单相
            \item pulsate vi.脉动
            \item three-phase power 三相电源
            \item three-phase circiut 三相电路
            \item the parallelogram method 平行四边形法则
            \item wye connection 星形连接
            \item delta connection 三角形连接
            \item phase voltage 相电压
            \item line voltage 线电压
            \item confuse vt.使混乱,使更难于理解,使困窘,使困惑 vi.使糊涂
            \item voltmeter n.电压表
            \item ammeter n.电流表
            \item clamp-on ammeter 钳式安培表
        \end{itemize}
    \end{multicols}

    \subsection{Further Reading}

    \section{Power Electronics}
    \subsection{Introduction}
    \begin{multicols}{2}
        \begin{itemize}
            \item solid-state adj.固态的
            \item computation n.计算,估计
            \item integration n.结合,整合,一体化
            \item dynamic adj.动态的,充满活力的,不断变化的 n.动态,动力学,活力
            \item mercury-arc [医]汞弧 \\Mercury n.水星
            \item valve n.阀,真空管,活栓 vt.装阀(于),以活门调节
            \item semiconductor n.[物理]半导体
            \item switching n.开关,转换,交换,配电 v.转换
            \item diode n.二极管
            \item inverter n.逆变器(反用换流器),变极器
            \item thyristor n.半导体闸流管,硅可控整流器
            \item inverter thyristor 晶体管逆变器,可控硅环流器,可控硅逆变器
            \item transistor n.晶体管,晶体管收音机,半导体收音机
            \item transmission n.播送,传送,信息,传动装置
            \item substantial adj.结实的,牢固的,重大的 n.本质,重要材料
            \item fluorescent lamp ballast 荧光灯镇流器
            \item mercury n.水银,汞
            \item therminoic adj.[物]热电子的,热离子的
            \item HVDC transmission system 高压直流输电系统
            \item induction motor 感应电动机
            \item vacuum n.真空,空间 adj.真空的 vt.用真空吸尘器打扫
            \item dissipate v.驱散,使(云/雾/疑虑)消散
            \item rectifier n.纠正者,整顿者,整流器
            \item triggered adj.触发的
            \item thyratron n.[电]闸流管
            \item ignitron n.引燃管,放电管
            \item cycloconverter n.周波变换器,循环换流器,双向离子变频器
            \item traic n.[电]三端双向可控硅开关元件
            \item a scope of 一个范围
            \item spectrum n.谱,光谱,范围,系列,幅度
        \end{itemize}
    \end{multicols}

    \ifx\collections\undefined
    \printbibliography %生成参考文献排版。
    \addcontentsline{toc}{chapter}{参考文献} %添加参考文献进目录
    \clearpage %新建页,确保超链接跳转正确
    \phantomsection %确保目录中的超链接指向正确的页码
    \printindex %生成索引排版。
\end{document}
    \fi
