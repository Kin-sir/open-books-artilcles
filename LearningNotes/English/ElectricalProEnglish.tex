\chapter{电力专业英语}
\section{Fundamentals of Electronic Circuits}
\subsection{Circuit Theory}
\begin{multicols}{2}
    \begin{itemize}
        \item element n.成分,元件
        \item interconnect vt.使互相连接
        \item node n.节点
        \item branch n.树枝,分枝,分部,支流,支脉 v.出现分岐
        \item loop n.环,循环,线圈(绳),弯曲部分,回路,回线 vt.使成环,使成圈,以环相连 vi.打环,翻筋斗
        \item topology n.拓扑,布局,拓扑学
        \item configuration n.构造,结构,配置,外形
        \item terminal n.终点站,终端,接线端
        \item resistor n.[电]电阻器
        \item independent n.独立自主者,无党派者 adj.独立自主的,不受约束的
        \item series n.连续,系列,级数,串联
        \item parallel adj.平行的,相同的,类似的,并联的 n.平行线,平行面,相似物 v.相应,平行
        \item impedance n.[电/物]阻抗,[物]全电阻
        \item theorem n.定理,法则(数)
    \end{itemize}
\end{multicols}

\subsection{Analog and Digital Circuits}
\begin{multicols}{2}
    \begin{itemize}
        \item analog n.(AmE)类似物;模拟 adj.(AmE)模拟的,指针式的\\analogue n./adj.(BrE)同上
        \item digital n.数字,数字式 adj.数字的,数位的
        \item thermometer n.温度计,体温计
        \item Fahrenheit n.华氏温度计 adj.华氏温度计的
        \item drum n.鼓,鼓声,鼓形圆筒 vt.打鼓奏 vi.击鼓,做鼓声
        \item discrete adj.不连续的,离散的
        \item original adj.最初的,原始的,独创的,新颖的 n.原物,原作
        \item remote adj.遥远的,偏僻的,细微的
        \item bulb n.鳞茎,球形物
        \item Morse code 莫尔斯电码
        \item pulse n.脉搏,脉冲
        \item buzzer n.蜂鸣器,信号手,嗡嗡作声的东西
        \item manipulate vt.(熟练地)操作,使用(机器),操纵(人或市场、市场),利用,应付
        \item destination n.目的地,[计]目的文件,目的单元格
        \item humidity n.湿气,潮湿,湿度
        \item comparator n.比较仪
        \item trigger vt.引发,引起,触发 vi.转义,换车 n.扳机
        \item sequence n.次序,顺序,序列
        \item parallel n.平行的,相同的,类似的,并联的 n.平行线,平行面,相似物 v.相应,平行
        \item serial adj.连续的,串行的,顺次
        \item decoder n.解码器
        \item reassemble vt.重新召集 vi.重新集合
    \end{itemize}
\end{multicols}

\subsection{Three-Phase Circuits}
\begin{multicols}{2}
    \begin{itemize}
        \item transformer n.变压器
        \item single-phase 单相
        \item pulsate vi.脉动
        \item three-phase power 三相电源
        \item three-phase circiut 三相电路
        \item the parallelogram method 平行四边形法则
        \item wye connection 星形连接
        \item delta connection 三角形连接
        \item phase voltage 相电压
        \item line voltage 线电压
        \item confuse vt.使混乱,使更难于理解,使困窘,使困惑 vi.使糊涂
        \item voltmeter n.电压表
        \item ammeter n.电流表
        \item clamp-on ammeter 钳式安培表
    \end{itemize}
\end{multicols}

\subsection{Further Reading}

\section{Power Electronics}
\subsection{Introduction}
\begin{multicols}{2}
    \begin{itemize}
        \item solid-state adj.固态的
        \item computation n.计算,估计
        \item integration n.结合,整合,一体化
        \item dynamic adj.动态的,充满活力的,不断变化的 n.动态,动力学,活力
        \item mercury-arc [医]汞弧 \\Mercury n.水星
        \item valve n.阀,真空管,活栓 vt.装阀(于),以活门调节
        \item semiconductor n.[物理]半导体
        \item switching n.开关,转换,交换,配电 v.转换
        \item diode n.二极管
        \item inverter n.逆变器(反用换流器),变极器
        \item thyristor n.半导体闸流管,硅可控整流器
        \item inverter thyristor 晶体管逆变器,可控硅环流器,可控硅逆变器
        \item transistor n.晶体管,晶体管收音机,半导体收音机
        \item transmission n.播送,传送,信息,传动装置
        \item substantial adj.结实的,牢固的,重大的 n.本质,重要材料
        \item fluorescent lamp ballast 荧光灯镇流器
        \item mercury n.水银,汞
        \item therminoic adj.[物]热电子的,热离子的
        \item HVDC transmission system 高压直流输电系统
        \item induction motor 感应电动机
        \item vacuum n.真空,空间 adj.真空的 vt.用真空吸尘器打扫
        \item dissipate v.驱散,使(云/雾/疑虑)消散
        \item rectifier n.纠正者,整顿者,整流器
        \item triggered adj.触发的
        \item thyratron n.[电]闸流管
        \item ignitron n.引燃管,放电管
        \item cycloconverter n.周波变换器,循环换流器,双向离子变频器
        \item traic n.[电]三端双向可控硅开关元件
        \item a scope of 一个范围
        \item spectrum n.谱,光谱,范围,系列,幅度
    \end{itemize}
\end{multicols}
