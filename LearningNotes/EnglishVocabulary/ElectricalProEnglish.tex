% !TEX root = main.tex %指定主文件
\ifx\collections\undefined
% !TEX program = xelatex %指定编译方式xelatex。
% !BIB program = biber %指定bib数据后台处理程序biber。
\documentclass[11pt,a4paper,UTF8,titlepage]{ctexart} %指定ctex的文档类,设置基本字号为11pt,a4大小,使用UTF-8编码保存,指定\maketitle生成单独的标题页。

\usepackage{syntonly}
%\syntaxonly %用来快速编译以排查错误,不生成DVI或PDF。
\usepackage[style=gb7714-2015]{biblatex} %调用biblatex宏包,设置参考文献样式(符合中文文献著录标准GB/T 7714-2015的样式),使用默认的后端程序biber(其支持更好,包括UTF-8等)放弃使用[backend=bibtex](只支持ascii编码)。
\addbibresource{resources/reference.bib} %加载参考文献数据库。
\usepackage{makeidx}
\makeindex %开启索引收集
\usepackage[margin=1in]{geometry} %调用geometry宏包,设置周围页边距为1英寸(为电子档设计,非打印)。
\usepackage{xcolor} %调用xcolor宏包,以支持扩展生成颜色。
\usepackage{fontspec} %调用fontspec宏包以更改西文字体族。
\setmainfont{Source Serif Pro}
\setsansfont{Source Sans Pro}
\setmonofont{Source Code Pro}
\usepackage{xeCJK} %调用xeCJK宏包以更改中文字体族。
\xeCJKsetup{AutoFakeSlant=true}  %设置xeCJK选项-伪斜体(需置于字体设置之前)。
\setCJKmainfont{思源宋体}
\setCJKsansfont{思源黑体}
\setCJKmonofont{思源等宽}
\usepackage{graphicx} %调用graphicx宏包,以支持插图。
\graphicspath{{resources/images/},{resources/images/follow/}} %加载图片路径。
\usepackage{caption} %调用caption,支持不带编号的标题。
\usepackage{wrapfig} %调用wrapfig,支持图文排版。
\usepackage{subfigure} %调用subfigure宏包进行图片图片排版。
\usepackage{tikz,tikz-qtree} %调用tikz及其扩展宏包,以支持画图。
\usepackage[subfigure]{tocloft} %tocloft与subfigure宏包冲突,不能简单调用,tocloft需设置参数。
\usepackage{tocbibind} %调用宏包以添加目录本身和参考文献进目录中。
\usepackage{multicol} %调用multicol宏包以支持多栏排版。
\usepackage[toc]{multitoc} %调用multitoc宏包,设置toc目录页,默认双栏排版。
\usepackage{enumitem} %调用emuitem,以设置列表环境。
%重订制目录命令。
\renewcommand{\tableofcontents}%
  {\chapter{\contentsname}%
  \@mkboth{\MakeUppercase\contentsname}{\MakeUppercase\contentsname}%
  \@makeschapterhead{\sourcecodename}%
  \@starttoc{toc}%
}
\usepackage{fancyhdr} %调用宏包,以设置页眉,统一格式:标题在左,页码在右。
\pagestyle{fancy}
\fancyhf{}
\fancyhead[LO]{\sffamily \rightmark}
\fancyhead[LE]{\sffamily \leftmark}
\fancyhead[ROE]{\bfseries \thepage}
%定义intro环境。
\newenvironment{intro}{\narrower\sffamily}{\par\vspace*{2ex plus 2.5ex minus 1.5ex}}
\usepackage{hyperref} %调用hyperref宏包
\hypersetup{
    colorlinks=true, %设置超链接文件带颜色
    bookmarks=true, %生成书签
    bookmarksopen=true, %书签展开
    bookmarksnumbered=true, %书签带章节编号
    CJKbookmarks=true, %cjk必设参数
    unicode, %utf-8编码必设参数
    pdftitle=标题, %设置PDF文件属性标题
    pdfauthor=Mr. Kin, %设置PDF文件属性作者
    pdfstartview=FitH %默认适合宽度显示
}

    \title{\hypertarget{title}{\textbf{标题}}}
    \addcontentsline{toc}{section}{标题页}
    \author{Written by Mr. Kin}
    \date{创建于2020.1.28,修改于\number\year.\number\month.\number\day}

\begin{document}
    \phantomsection %确保目录中的超链接指向正确的页码
    \pdfbookmark[1]{标题页}{title}
    \maketitle %生成标题页
    {\centering \tableofcontents} %生成目录页
    \clearpage
    \fi

    %\pagenumbering{arabic} %阿拉伯样式页码,非第一个分文件请注释此行。
    \section{电力专业英语}
    \subsection{Fundamentals of Electronic Circuits}
    \subsubsection{Circuit Theory}
    \begin{multicols}{2}
        \begin{itemize}[left=0em]
            \item element n.成分,元件
            \item interconnect vt.使互相连接
            \item node n.节点
            \item branch n.树枝,分枝,分部,支流,支脉 v.出现分岐
            \item loop n.环,循环,线圈(绳),弯曲部分,回路,回线 vt.使成环,使成圈,以环相连 vi.打环,翻筋斗
            \item topology n.拓扑,布局,拓扑学
            \item configuration n.构造,结构,配置,外形
            \item terminal n.终点站,终端,接线端
            \item resistor n.电阻器(电)
            \item independent n.独立自主者,无党派者 adj.独立自主的,不受约束的
            \item series n.连续,系列,级数,串联
            \item parrallel adj.平行的,相同的,类似的,并联的 n.平行线,平行面,相似物 v.相应,平行
            \item impedance n.阻抗,全电阻(电/物)
            \item theorem n.定理,法则(数)
        \end{itemize}
    \end{multicols}

    \ifx\collections\undefined
    \printbibliography %生成参考文献排版。
    \printindex %生成索引排版。
\end{document}
    \fi