\section{实战篇}

\subsection{SVG转3D网格}
\begin{enumerate}
    \item 切换到顶视图-7键。
    \item 菜单>文件>导入>SVG
    \item 缩放视图找到SVG。
    \item 右键set origin,调整“原点”至物体中心。
    \item shift+s键对齐“原点”至世界中心
    \item 选中并缩放至合适大小。
    \item 右键convert to,转换到Mesh。
    \item 选中网格物体,TAB键进入“编辑模式”。
    \item A键选中所有网格。
    \item 切换到侧视图-3或1键。(按住Ctrl时为视图反方向)
    \item E键挤出面。
    \item TAB键回到“物体模式”。
\end{enumerate}

\subsection{灰尘粒子动画}
无特殊说明,则都是在粒子标签内操作。
\begin{enumerate}
    \item 创建合适的立方体(粒子场)
    \item 选择立方体,属性编辑器>粒子标签>新建粒子系统
    \item 粒子类型>发射体;发射>源>发射源>体积
    \item 创建灰尘物体模型
    \item 渲染>渲染为>物体(此为单个效果,用集合来实现多个),物体>选择灰尘物体模型
    \item 渲染>缩放>数值调整;缩放随机性>1
    \item 1键切换到前视图,ctrl+alt+0并左击相机视角边框处以选中相机,按G调整相机视角;属性编辑器>相机属性>镜头>焦距>数值调整。
    \item 启用旋转功能>坐标系轴向>normal;随机>数值调整;相位>数值调整;随机相位>数值调整
    \item 选中立方体,属性编辑器>材质>表面>移除材质,材质>体积>原理化体积>密度调0.14
    \item 属性编辑器>世界属性>颜色调至黑色(除去环境亮度)
    \item
\end{enumerate}
